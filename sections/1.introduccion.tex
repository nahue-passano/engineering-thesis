\subsection{Fundamentación}

La evolución tecnológica y la adopción de nuevas tecnologías han marcado un punto de inflexión en la producción de contenido digital. Específicamente, la integración de la inteligencia artificial en la creación de imágenes, videos y audio ha desencadenado una revolución significativa en términos de calidad, técnicas y velocidad de generación de nuevo contenido. Ante este panorama, la necesidad de adaptar y mejorar continuamente los métodos de producción se ha vuelto imperativa. La creciente demanda de accesibilidad y personalización en los medios digitales impulsa la búsqueda de soluciones innovadoras que no solo optimicen los procesos existentes sino que también abran nuevas vías para la creación de material. 
En este contexto, los audiolibros emergen como un medio particularmente relevante, dada su creciente popularidad y su capacidad para ofrecer una experiencia complementaria e innovadora a la lectura tradicional. Sin embargo, el desarrollo de audiolibros presenta desafíos singulares. En primer lugar, se necesita contratar a un narrador capaz de cautivar con sus historias. Además, para ofrecer una experiencia más inmersiva, es necesario grabar en un entorno acústicamente controlado, utilizando equipos profesionales como micrófonos de alta fidelidad y una placa de audio de calidad. Por último, se requiere un trabajo de post-producción realizado por un especialista. Estos requisitos implican que la producción conlleva una inversión significativa, tanto en términos financieros (para la contratación del narrador y los profesionales encargados de la grabación) como en cuanto al tiempo necesario. Ante este panorama, se propone investigar el diseño e implementación de un sistema automatizado de generación de audiolibros, el cual se centra principalmente en producir narraciones naturales y de alta calidad. Este sistema busca abordar dichos desafíos mediante el aprovechamiento de las últimas innovaciones en inteligencia artificial aplicada al procesamiento digital de señales, representando un paso adelante significativo en el campo. La integración de estas tecnologías no solo permitirá una producción más eficiente, personalizable y rápida, sino que también mejorará la calidad del producto final, proporcionando una herramienta capaz de narrar textos con la fluidez de un locutor profesional. La generación de voz sintética, en particular, se basa en modelos de aprendizaje profundo que pueden capturar las inflexiones y emociones humanas, ofreciendo una experiencia auditiva agradable y reconfortante. Por otro lado, el procesamiento digital de señales juega un papel crucial en la mejora de la claridad y la calidad del habla, además de ser una herramienta fundamental en el preprocesamiento y postprocesamiento de los audios a utilizar. Otro desafío significativo en el presente desarrollo es la adaptación tecnológica al idioma español, específicamente al dialecto rioplatense. Aunque existen numerosos sistemas capaces de sintetizar discursos en un español nativo, la capacidad para abordar las diversas variantes y dialectos del español es limitada. Por lo tanto, esta investigación se centra en la creación de una herramienta que logre representar y narrar historias incorporando los elementos característicos del dialecto rioplatense, tales como el yeísmo, su particular ritmo y sus marcadores de identidad. Por último, se propone incorporar como narradores a figuras destacadas de la cultura argentina, con el objetivo de generar identificación e impacto en los potenciales usuarios de la plataforma.


\subsection{Objetivos}

\subsubsection{Objetivo general}

El objetivo principal de esta investigación es desarrollar una herramienta de acceso libre que permita la generación de audiolibros personalizables con voces reconocidas de la cultura  argentina.

\subsubsection{Objetivos específicos}

\begin{itemize}
    \item Recopilar discursos de figuras icónicas de Argentina, seleccionando las voces más distintivas y reconocidas de su cultura.
    \item Crear una plataforma de preprocesamiento de audio que facilite la obtención de oraciones y sus transcripciones automáticas.
    \item Conformar una base de datos idónea para el entrenamiento y ajuste del sistema con la plataforma diseñada.
    \item Implementar un sistema de validación para las transcripciones automatizadas, asegurando un control riguroso de la calidad de los audios y sus correspondientes transcripciones.
    \item Realizar un ajuste fino del modelo para emular las voces seleccionadas.
    \item Llevar a cabo encuestas subjetivas para evaluar la calidad, naturalidad y similitud de las narraciones generadas.
    \item Utilizar métricas objetivas que complementen las evaluaciones subjetivas recogidas.
    \item Comparar los resultados obtenidos con investigaciones previas para evaluar objetiva y subjetivamente el desempeño del sistema.
    \item Proveer una plataforma de acceso libre que permita el uso del servicio desarrollado.

\end{itemize}

\subsection{Estructura de la investigación}

En el capítulo 2 se presenta una revisión meticulosa del estado del arte y la construcción del marco teórico detallado. Este marco abarca estudios previos y desarrollos tecnológicos en áreas clave. Se destacan tanto las soluciones existentes como las limitaciones actuales, estableciendo un contexto para la innovación propuesta y asegurando que el diseño e implementación del sistema consideren las mejores prácticas y los hallazgos más recientes. 

En el capítulo 3 se describe el paso a paso en el desarrollo de la herramienta, detallando etapas claves como la recolección y conformación del conjunto de datos a utilizar, el entrenamiento y ajuste del sintetizador elegido y el diseño de la plataforma que será utilizada por los usuarios finales.

En el capítulo 4 se comenta como se evalúa el sistema desarrollado, describiendo tanto su evaluación objetiva como subjetiva. En el mismo se detallan las métricas a utilizar, las cuales intentan cuantificar distintas características de los discursos sintéticos generados, como lo son la naturalidad, la calidad y la similitud con la voz objetivo. A su vez, se describe el proceso para evaluar el sistema desde una perspectiva subjetiva con una prueba del tipo Mean Opinion Score (MOS).

En el capítulo 5 se lleva a cabo una validación en términos estadísticos y metodológicos de la muestra recolectada en la prueba subjetiva, describiendo los métodos para su análisis y ...

En el capítulo 6 se realiza el análisis de los resultados, donde se nuclea los análisis realizados en los capítulos 4 y 5 en pos de tener una representación real del rendimiento del sistema (sanata).

En el capítulo 7 se realizan las conclusiones finales del sistema desarrollado, marcando los hitos de todo el desarrollo y los aspectos que pueden ser mejorados.

En el capitulo 8 se esbozan ideas y propuestas para líneas futuras de investigación, donde se intenta un panorama general de las mejoras del sistema, así como la dirección de investigación en el estado del arte actual. (sanata)
