En esta sección, se establecen las variables necesarias para la evaluación objetiva y subjetiva del sistema. En particular, se enfoca en medir el desempeño y valorar la percepción de los discursos generados por el sistema.

\subsection{Evaluación objetiva}



\subsection{Evaluación subjetiva}

Para realizar la evaluación subjetiva se plantea realizar una encuesta del tipo Mean Opinion Score (MOS) (ITU-T). El MOS es un procedimiento que permite obtener percepciones subjetivas sobre la calidad (MOS), similitud (Sim-MOS) y naturalidad (Nat-MOS) de las voces sintéticas generadas. Para llevar a cabo esta evaluación, se seleccionan diez muestras de audio generadas por el sistema, correspondientes a diez oradores diferentes, y se presentan a un grupo de oyentes. Se procura que la composición del grupo mantenga un equilibrio entre oyentes expertos en tópicos de audio y oyentes generales. A fin de que la muestra de participantes sea significativa y representativa, se necesita contar con al menos veinticinco participantes. Cada participante escucha estas muestras y se solicita calificarlas en una escala del 1 al 5 en función de tres criterios clave: calidad, similitud y naturalidad. La calidad se refiere a cuán bien se percibe la fidelidad y la claridad del audio sintetizado. La similitud evalúa cuán cercana es la voz sintética a la voz original del hablante de referencia. La naturalidad se refiere a cuán fluido y humano suena el discurso sintetizado. Es importante destacar que el orden de presentación de las muestras se debe organizar de manera aleatoria para cada participante en pos de evitar sesgos potenciales. Además, cada muestra de audio se debe escuchar una sola vez, garantizando que las calificaciones se basan en las impresiones inmediatas de los participantes. Por último se realiza una pregunta sobre la experiencia del participante en escucha crítica y/o uso de sistemas de TTS.
