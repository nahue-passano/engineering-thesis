La validación de las pruebas comienza una vez recolectados los datos de la encuesta subjetiva. Para cada respuesta, se calcula el MOS, el Nat-MOS y el Sim-MOS promediando las respuestas para cada par de audios presentados. Luego se descartan las respuestas anómalas (también llamadas outliers) para poder realizar un análisis de normalidad y homocedasticidad. Dichas pruebas se realizan en Python con la librería statsmodels (Seabold \& Perktold, 2010).